\documentclass[]{article}
\usepackage{lmodern}
\usepackage{amssymb,amsmath}
\usepackage{ifxetex,ifluatex}
\usepackage{fixltx2e} % provides \textsubscript
\ifnum 0\ifxetex 1\fi\ifluatex 1\fi=0 % if pdftex
  \usepackage[T1]{fontenc}
  \usepackage[utf8]{inputenc}
\else % if luatex or xelatex
  \ifxetex
    \usepackage{mathspec}
  \else
    \usepackage{fontspec}
  \fi
  \defaultfontfeatures{Ligatures=TeX,Scale=MatchLowercase}
\fi
% use upquote if available, for straight quotes in verbatim environments
\IfFileExists{upquote.sty}{\usepackage{upquote}}{}
% use microtype if available
\IfFileExists{microtype.sty}{%
\usepackage{microtype}
\UseMicrotypeSet[protrusion]{basicmath} % disable protrusion for tt fonts
}{}
\usepackage[margin=1in]{geometry}
\usepackage{hyperref}
\hypersetup{unicode=true,
            pdftitle={Análise do impacto da expansão multicampi das universidades federais na dinâmica econômica e social dos municípios paraenses utilizando Data Analytics.},
            pdfauthor={Camacho, M.S.},
            pdfborder={0 0 0},
            breaklinks=true}
\urlstyle{same}  % don't use monospace font for urls
\usepackage{color}
\usepackage{fancyvrb}
\newcommand{\VerbBar}{|}
\newcommand{\VERB}{\Verb[commandchars=\\\{\}]}
\DefineVerbatimEnvironment{Highlighting}{Verbatim}{commandchars=\\\{\}}
% Add ',fontsize=\small' for more characters per line
\usepackage{framed}
\definecolor{shadecolor}{RGB}{248,248,248}
\newenvironment{Shaded}{\begin{snugshade}}{\end{snugshade}}
\newcommand{\AlertTok}[1]{\textcolor[rgb]{0.94,0.16,0.16}{#1}}
\newcommand{\AnnotationTok}[1]{\textcolor[rgb]{0.56,0.35,0.01}{\textbf{\textit{#1}}}}
\newcommand{\AttributeTok}[1]{\textcolor[rgb]{0.77,0.63,0.00}{#1}}
\newcommand{\BaseNTok}[1]{\textcolor[rgb]{0.00,0.00,0.81}{#1}}
\newcommand{\BuiltInTok}[1]{#1}
\newcommand{\CharTok}[1]{\textcolor[rgb]{0.31,0.60,0.02}{#1}}
\newcommand{\CommentTok}[1]{\textcolor[rgb]{0.56,0.35,0.01}{\textit{#1}}}
\newcommand{\CommentVarTok}[1]{\textcolor[rgb]{0.56,0.35,0.01}{\textbf{\textit{#1}}}}
\newcommand{\ConstantTok}[1]{\textcolor[rgb]{0.00,0.00,0.00}{#1}}
\newcommand{\ControlFlowTok}[1]{\textcolor[rgb]{0.13,0.29,0.53}{\textbf{#1}}}
\newcommand{\DataTypeTok}[1]{\textcolor[rgb]{0.13,0.29,0.53}{#1}}
\newcommand{\DecValTok}[1]{\textcolor[rgb]{0.00,0.00,0.81}{#1}}
\newcommand{\DocumentationTok}[1]{\textcolor[rgb]{0.56,0.35,0.01}{\textbf{\textit{#1}}}}
\newcommand{\ErrorTok}[1]{\textcolor[rgb]{0.64,0.00,0.00}{\textbf{#1}}}
\newcommand{\ExtensionTok}[1]{#1}
\newcommand{\FloatTok}[1]{\textcolor[rgb]{0.00,0.00,0.81}{#1}}
\newcommand{\FunctionTok}[1]{\textcolor[rgb]{0.00,0.00,0.00}{#1}}
\newcommand{\ImportTok}[1]{#1}
\newcommand{\InformationTok}[1]{\textcolor[rgb]{0.56,0.35,0.01}{\textbf{\textit{#1}}}}
\newcommand{\KeywordTok}[1]{\textcolor[rgb]{0.13,0.29,0.53}{\textbf{#1}}}
\newcommand{\NormalTok}[1]{#1}
\newcommand{\OperatorTok}[1]{\textcolor[rgb]{0.81,0.36,0.00}{\textbf{#1}}}
\newcommand{\OtherTok}[1]{\textcolor[rgb]{0.56,0.35,0.01}{#1}}
\newcommand{\PreprocessorTok}[1]{\textcolor[rgb]{0.56,0.35,0.01}{\textit{#1}}}
\newcommand{\RegionMarkerTok}[1]{#1}
\newcommand{\SpecialCharTok}[1]{\textcolor[rgb]{0.00,0.00,0.00}{#1}}
\newcommand{\SpecialStringTok}[1]{\textcolor[rgb]{0.31,0.60,0.02}{#1}}
\newcommand{\StringTok}[1]{\textcolor[rgb]{0.31,0.60,0.02}{#1}}
\newcommand{\VariableTok}[1]{\textcolor[rgb]{0.00,0.00,0.00}{#1}}
\newcommand{\VerbatimStringTok}[1]{\textcolor[rgb]{0.31,0.60,0.02}{#1}}
\newcommand{\WarningTok}[1]{\textcolor[rgb]{0.56,0.35,0.01}{\textbf{\textit{#1}}}}
\usepackage{graphicx,grffile}
\makeatletter
\def\maxwidth{\ifdim\Gin@nat@width>\linewidth\linewidth\else\Gin@nat@width\fi}
\def\maxheight{\ifdim\Gin@nat@height>\textheight\textheight\else\Gin@nat@height\fi}
\makeatother
% Scale images if necessary, so that they will not overflow the page
% margins by default, and it is still possible to overwrite the defaults
% using explicit options in \includegraphics[width, height, ...]{}
\setkeys{Gin}{width=\maxwidth,height=\maxheight,keepaspectratio}
\IfFileExists{parskip.sty}{%
\usepackage{parskip}
}{% else
\setlength{\parindent}{0pt}
\setlength{\parskip}{6pt plus 2pt minus 1pt}
}
\setlength{\emergencystretch}{3em}  % prevent overfull lines
\providecommand{\tightlist}{%
  \setlength{\itemsep}{0pt}\setlength{\parskip}{0pt}}
\setcounter{secnumdepth}{0}
% Redefines (sub)paragraphs to behave more like sections
\ifx\paragraph\undefined\else
\let\oldparagraph\paragraph
\renewcommand{\paragraph}[1]{\oldparagraph{#1}\mbox{}}
\fi
\ifx\subparagraph\undefined\else
\let\oldsubparagraph\subparagraph
\renewcommand{\subparagraph}[1]{\oldsubparagraph{#1}\mbox{}}
\fi

%%% Use protect on footnotes to avoid problems with footnotes in titles
\let\rmarkdownfootnote\footnote%
\def\footnote{\protect\rmarkdownfootnote}

%%% Change title format to be more compact
\usepackage{titling}

% Create subtitle command for use in maketitle
\newcommand{\subtitle}[1]{
  \posttitle{
    \begin{center}\large#1\end{center}
    }
}

\setlength{\droptitle}{-2em}

  \title{Análise do impacto da expansão multicampi das universidades federais na
dinâmica econômica e social dos municípios paraenses utilizando Data
Analytics.}
    \pretitle{\vspace{\droptitle}\centering\huge}
  \posttitle{\par}
    \author{Camacho, M.S.}
    \preauthor{\centering\large\emph}
  \postauthor{\par}
    \date{}
    \predate{}\postdate{}
  

\begin{document}
\maketitle

\hypertarget{introducao}{%
\subsection{Introdução}\label{introducao}}

\hypertarget{resumo}{%
\subsubsection{RESUMO}\label{resumo}}

O projeto propõe a investigação do impacto da Universidade Federal no
ambiente e economia municipal e sua contribuição para o desenvolvimento
sustentável e melhoria de índices sociais, humanos e demográficos,
calcados na hipótese de que os processos de seleção acadêmica e formação
de quadros profissionais na dinâmica de expansão universitária figuram
como coeficientes de desenvolvimento. Tais análises serão construídas
com alunos que se interessam pela área da estatística e programação, bem
como para aqueles que têm curiosidade em ver aplicados os conceitos
estudados em disciplinas de estatística, economia e programação. Durante
a jornada, serão abordados conceitos de bancos de dados: aquisição,
tratamento e carga de dados, consultas analíticas e exportação de dados;
Planilhas eletrônicas: tabelas dinâmicas, gráficos, estatísticas
descritivas e ajuste de curvas; e a ferramenta GNU R com todo seu
potencial de análise com aplicação de técnicas de Data Mining e
visualização de dados. Conhecimentos que serão extremamente úteis na
jornada acadêmica e profissional.

\hypertarget{justificativa}{%
\subsubsection{JUSTIFICATIVA}\label{justificativa}}

A expansão da educação superior está associada diretamente com a
formação de mão de obra qualificada para o mercado de trabalho e o
fortalecimento do setor de pesquisa e desenvolvimento, com cursos de
graduação e pós-graduação, no entanto, o processo de criação de
universidades e campi universitários também contribui com a economia
local, sobretudo em pequenos municípios. Fora da dimensão educacional,
pouco se sabe sobre o impacto da política multicampi, ou sobre o
processo de interiorização da Universidade Federal do Pará e como ela
alterou a dimensão demográfica, econômica, social e cultural dos
municípios que as acolheram. Conhecer as consequências do esforço de
aumentar a presença institucional nas comunidades paraenses é um
exercício de autoconhecimento que deve ser praticado para agregar valor
às universidades públicas e aos municípios que apoiam tal processo. LUZ
et al. (2017) destacam o papel das Universidades Públicas no
desenvolvimento sustentável, caracterizando-as como instituições de
ensino, pesquisa e extensão, mas também culturais e humanas. Dados
estatísticos do Censo da Educação Superior de 2016 mostram o galopante
crescimento de instituições privadas, com acentuada presença em capitais
e em ambientes EAD. No entanto, quando se trata apenas das ofertas de
cursos presenciais, as instituições particulares ofertaram 56\% dos seus
cursos no Interior conseguindo um saldo de 44\% das matrículas, enquanto
que a oferta de cursos fora da capital por instituições públicas foi de
76\%, com 52\% das matrículas, mostrando que a capilaridade e a força
das instituições Públicas de Ensino Superior ainda é mais forte. O
processo de seleção e oferta de cursos especializados fora da capital é
um diferencial para as universidades públicas, o SISU (Sistema de
Seleção Unificada) tem permitido uma maior distribuição das vagas,
oferecendo ao candidato a oportunidade de ingressar no ensino superior
em instituições sediadas em outros municípios. Essa migração é positiva
no sentido de reduzir a endogenia na formação de quadros profissionais,
no entanto, como demonstrado por LI (2016), o SISU aumenta a
probabilidade de migração interestadual em 3,9\%. Tal processo de
migração, pode provocar efeitos sensíveis na dinâmica de pequenos
municípios: seja no turismo, habitação, PIB, entre outros índices
sociais e demográficos. A ampliação da rede federal de ensino com a
política de interiorização multicampi, da qual a UFPA é pioneira,
propiciou em vários casos, colaborações exitosas entre a universidade e
a administração pública municipal, no sentido de apoiar o interesse do
desenvolvimento local com a oferta de cursos específicos, ou mesmo na
oferta de qualificação para agentes municipais. No ponto de vista
microeconômico, verifica-se a participação do valor adicionado bruto a
preços correntes da administração, defesa, educação e saúde públicas e
seguridade social representa até 29\% do PIB para municípios paraenses,
representando na sua maioria a força de trabalho da administração
pública municipal. A criação de um órgão com quadros super qualificados,
como é o caso de campi universitários, que além de professores doutores,
oferece vagas para assistentes e técnicos de nível superior e médio
aumenta a renda municipal, e consequentemente as transferências
sucessivas de renda . A ciência de dados está se tornando a atividade
mais popular do século, inclusive ciências cuja natureza de análise eram
predominantemente qualitativas, incluem em seus currículos disciplinas
matemáticas, estatísticas e de computação estudando métodos analíticos
ou heurísticos de análise de fenômenos (DAVENPORT, 2012). Em várias
situações acadêmicas e profissionais é exigido familiaridade com dados:
atividades experimentais, controle de processos, atividades de
investigação ou ensaios laboratoriais etc. todas essas atividades
tornam-se simples se operadas com as ferramentas certas e conhecimentos
básicos em ciência de dados. Fica cada vez mais claro, que tais
habilidades são esperadas de profissionais do futuro, profissionais que
lidarão com Big Data, analisando dados provenientes de redes sociais e
bases de dados estruturadas ou não. Além disso, ter a capacidade de
operar grandes volumes de dados não é suficiente, é necessário ter o
conhecimento certo para analisá-los e extrair deles informações úteis.
Árvores de decisão, redes bayesianas, redes neurais, KNN, Cluster etc
são exemplos de técnicas aplicadas comumente em Big Data, e devem ser
familiares à geração de estudantes que ingressarão em um ambiente
profissional cada vez mais exigente e competitivo.

\hypertarget{objetivos}{%
\subsubsection{OBJETIVOS}\label{objetivos}}

\begin{itemize}
\tightlist
\item
  Conhecer o impacto da Universidade Federal do Pará na economia
  municipal;
\item
  Explorar conceitos de estatística e análise de dados com alunos do
  Campus de Salinópolis, aplicando conceitos de estatística e
  programação explorados durante o curso;
\item
  Popularizar ferramentas de bancos de dados, estatística e Data Mining;
\item
  Oferecer às instituições locais indicadores sociais em de panoramas
  analíticos e infográficos, subsidiando a tomada de decisões, sobretudo
  para gestores públicos.
\end{itemize}

\hypertarget{metas}{%
\subsubsection{METAS}\label{metas}}

\begin{itemize}
\tightlist
\item
  Construir uma base de dados estratificada por município com as
  variáveis para estudo de fenômenos sociais, culturais, ambientais,
  educacionais, econômicos e demográficos para pesquisadores e alunos.
\item
  Publicação de artigos em congresso/conferência nacional sobre o estudo
  do impacto das universidades federais paraenses;
\item
  Oferecer mini-cursos sobre bases de dados e ferramentas de Data
  Analytics.
\end{itemize}

\hypertarget{metodologia}{%
\subsubsection{METODOLOGIA}\label{metodologia}}

Utilizando fontes de dados abertas e públicas, e meios de acesso à
informação como Sistemas de Transparência Governamentais será possível
agregar bases históricas de variáveis indicadores sociais, demográficos
e econômicos estratificados por município. Após a construção do Banco de
Dados utilizando o SGBD PostgreSQL, será criado um DataWarehouse
utilizando a engine Pentaho para a implantação de uma ferramenta OLAP de
Business Intelligence, permitindo aos usuários a criação de estatísticas
descritivas e gráficos online, combinando uma ou mais dimensões com as
variáveis de interesse. Tais dados, poderão ser dinamicamente
transformados e exportados para uma ou mais ferramentas estatísticas,
como planilhas eletrônicas ou mesmo softwares estatísticos, como o GNU
R. Utilizando o software estatístico GNU R, serão aplicadas técnicas de
aprendizado de máquina não supervisionados, como Clusterização e KNN,
objetivando identificar padrões entre municípios que contam com a
presença de um Campi Universitário. Já as técnicas de aprendizado
supervisionado, como árvore de decisão e redes bayesianas serão
utilizadas para compreender melhor o fenômeno, de modo a enriquecer a
análise da estatística descritiva. Por fim, serão realizados testes
estatísticos de média e variância para validar as hipóteses levantadas.
Todas as ferramentas e conhecimentos necessários serão oferecidos aos
alunos do Campus, principalmente aos discentes vinculados ao curso de
matemática. O aluno aprenderá em um minicurso prático os principais
conhecimentos necessários para criação e implantação de bases de dados,
operações e planejamento. Também será oferecido um minicurso sobre o
software livre GNU R, que figura entre os principais softwares
destinados à Data Analytics. Por fim, os resultados devem ser
apresentados para a comunidade acadêmica em eventos de extensão, como
também popularizado pela comunidade científica através de artigos e
apresentação em congressos científicos. Durante o intercurso do projeto
serão oferecidos minicursos práticos, focando a temática do projeto e
utilizando dados reais. No minicurso de Planilhas de texto e tratamento
de dados os alunos serão familiarizados com operações de busca, edição,
retabulação de dados, expressões regulares, estatísticas descritivas,
fórmulas e índices, gráficos e análise de regressão para ajuste de
curvas. No minicurso de Banco de dados, os alunos aprenderão sobre o
processo de criação e manutenção de um banco de dados relacional:
criação/edição/deleção de tabelas e inserção/busca/deleção/alteração de
dados, utilizando o SGBD PostgreSQL. Por fim, será apresentada a
ferramenta GNU R e o ambiente RStudio, onde serão estudados os tipos de
dados e serão implementadas funções estatísticas para análise dos dados
e geração de gráficos poderosos e elegantes.

\hypertarget{carga}{%
\subsection{Carga}\label{carga}}

Os dados brutos foram obtidos do Instituto Brasileiro de Geografia e
Estatística (IBGE), através do link
(\url{https://www.ibge.gov.br/estatisticas/economicas/contas-nacionais/9088-produto-interno-bruto-dos-municipios.html?t=resultados}),
menu `'Download'', disponibilizados em dois arquivos: Base 2010-2016,
que utiliza a metodologia corrente de estimação do PIB, e outro arquivo
nominado de Base 2002-2009, que são indicadores estimados por
retropolação. Os arquivos são disponibilizados em XLS, ODS e TXT, e
transmitidos compactados para download. Para mais detalhes sobre a
metodologia de estimação do PIB a preços correntes, pode ser consultado
a documentação disponível em:

\begin{itemize}
\item
  Retropolação -
  \url{ftp://ftp.ibge.gov.br/Pib_Municipios/Notas_Metodologicas_2010/NotaMetodologicaPIB_MunicipiosRetropolacao.pdf}
\item
  Á partir de 2010 -
  \url{ftp://ftp.ibge.gov.br/Pib_Municipios/Notas_Metodologicas_2010/NotaMetodologicaPIB_MunicipiosRef2010.pdf}
\end{itemize}

\begin{Shaded}
\begin{Highlighting}[]
\KeywordTok{library}\NormalTok{(readxl)}
\NormalTok{  dados02a09 <-}\StringTok{ }\KeywordTok{read_excel}\NormalTok{(}\StringTok{'../DADOS/PIB dos Municipios - base de dados 2002-2009.xls'}\NormalTok{)}
\NormalTok{  dados10a16 <-}\StringTok{ }\KeywordTok{read_excel}\NormalTok{(}\StringTok{'../DADOS/PIB dos Municipios - base de dados 2010-2016.xls'}\NormalTok{)}
\end{Highlighting}
\end{Shaded}

Percebemos que o arquivo da série 2010 a 2016 apresenta 5 variáveis
extras:

\begin{Shaded}
\begin{Highlighting}[]
\KeywordTok{names}\NormalTok{(dados10a16)[}\KeywordTok{names}\NormalTok{(dados10a16) }\OperatorTok\StringTok{ }\KeywordTok{names}\NormalTok{(dados02a09)}\OperatorTok{==}\StringTok{"FALSE"}\NormalTok{]}
\end{Highlighting}
\end{Shaded}

\begin{verbatim}
## [1] "População\n(Nº de habitantes)"                      
## [2] "Produto Interno Bruto per capita\n(R$ 1,00)"        
## [3] "Atividade com maior valor adicionado bruto"         
## [4] "Atividade com segundo maior valor adicionado bruto" 
## [5] "Atividade com terceiro maior valor adicionado bruto"
\end{verbatim}

\hypertarget{transformacao}{%
\subsection{Transformação}\label{transformacao}}

\hypertarget{renominacao-de-fatores}{%
\subsubsection{Renominação de fatores}\label{renominacao-de-fatores}}

\begin{Shaded}
\begin{Highlighting}[]
\NormalTok{nomes02a09<-}\KeywordTok{c}\NormalTok{(}\StringTok{"ano"}\NormalTok{,}\StringTok{"cod_granderegiao"}\NormalTok{,}\StringTok{"nome_granderegiao"}\NormalTok{,}\StringTok{"cod_uf"}\NormalTok{,}\StringTok{"sgl_uf"}\NormalTok{,}\StringTok{"nome_uf"}\NormalTok{,}\StringTok{"cod_municipio"}\NormalTok{,}\StringTok{"nome_municipio"}\NormalTok{,}
              \StringTok{"regiao_metropolitana"}\NormalTok{,}\StringTok{"cod_mesoregiao"}\NormalTok{,}\StringTok{"nome_mesoregiao"}\NormalTok{,}\StringTok{"cod_microregiao"}\NormalTok{,}\StringTok{"nome_microregiao"}\NormalTok{,}
              \StringTok{"cod_regiaorural"}\NormalTok{,}\StringTok{"nome_regiaorural"}\NormalTok{,}\StringTok{"tipo_regiaorural"}\NormalTok{,}\StringTok{"cod_regiao_imediata"}\NormalTok{,}\StringTok{"nome_regiao_imediata"}\NormalTok{,}\StringTok{"mun_imediata"}\NormalTok{,}
              \StringTok{"cod_regiao_intermediaria"}\NormalTok{,}\StringTok{"nome_regiao_intermediaria"}\NormalTok{,}\StringTok{"mun_intermediaria"}\NormalTok{,}\StringTok{"amazonia_legal"}\NormalTok{,}\StringTok{"semiarido"}\NormalTok{,}\StringTok{"cod_concentracao_urbana"}\NormalTok{,}
              \StringTok{"nome_concentracao_urbana"}\NormalTok{,}\StringTok{"tipo_concentracao_urbana"}\NormalTok{,}\StringTok{"cod_arranjo_populacional"}\NormalTok{,}\StringTok{"nome_arranjo_populacional"}\NormalTok{,}\StringTok{"tipologia_rural_urbana"}\NormalTok{,}\StringTok{"hierarquia_urbana"}\NormalTok{,}\StringTok{"princ_cat_hier_urbana"}\NormalTok{,}\StringTok{"cid_reg_sp"}\NormalTok{,}\StringTok{"VAB_agropecuaria"}\NormalTok{,}\StringTok{"VAB_industria"}\NormalTok{,}\StringTok{"VAB_servicos"}\NormalTok{,}\StringTok{"VAB_administracao"}\NormalTok{,}\StringTok{"VAB_total"}\NormalTok{,}
              \StringTok{"impostos"}\NormalTok{,}\StringTok{"pib"}\NormalTok{)}

\NormalTok{nomes10a16<-}\KeywordTok{c}\NormalTok{(nomes02a09,}\StringTok{"populacao"}\NormalTok{,}\StringTok{"pib_percapita"}\NormalTok{,}\StringTok{"ativ_1_maior_vab"}\NormalTok{,}\StringTok{"ativ_2_maior_vab"}\NormalTok{,}\StringTok{"ativ_3_maior_vab"}\NormalTok{)}

\KeywordTok{names}\NormalTok{(dados02a09)<-nomes02a09}
\KeywordTok{names}\NormalTok{(dados10a16)<-nomes10a16}
\end{Highlighting}
\end{Shaded}

\hypertarget{analises}{%
\subsection{Análises}\label{analises}}


\end{document}
